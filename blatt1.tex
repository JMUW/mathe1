\documentclass{article}
\usepackage[utf8]{inputenc}

\title{Mathe für Informatiker 1}
\author{Vorname Name Matrikelnummer}
\begin{document}
\maketitle

\section{Übungsblatt 1}
\subsection{H1-1 Aussagenlogik}
\subsubsection{\((A \Rightarrow B) \iff (\neg A) \land B  \)}

Sei A und B wahr, so ist die Implikation \(A \Rightarrow B\) ebenfalls wahr.
Da aber \(\neg A \land B  \) falsch, so sind die Aussagen nicht tautologisch gleichwertig und somit keine Tautologie.

\subsubsection{\((A \Rightarrow B) \iff (\neg A) \lor B  \)}
\begin{center}
\begin{tabular}{||c c | c c | c ||}
  A & B & \(A \Rightarrow B\) & \((\neg A) \lor B\) &\(... \iff ...\) \\
  \hline
  w & w & w & w & w \\
  w & f & f & f & w \\
  f & w & w & w & w \\
  f & f & w & w & w \\
\end{tabular}
\end{center}
Gemäß Wahrheitstabelle stimmen die Aussagen \((A \Rightarrow B) \) und \( (\neg A) \lor B  \)  für alle Kombinationen aus A und B überein und sind somit Tautologien.

\subsubsection{\(((A \Rightarrow B) \land (\neg B)) \Rightarrow  (\neg A)  \)}

\begin{center}
\begin{tabular}{||c c | c c| c ||}
  A & B & \(((A \Rightarrow B) \land (\neg B))\) & \((\neg A)\)  & \(... \Rightarrow ...\) \\
  \hline
  w & w & f & f & w \\
  w & f & f & f & w \\
  f & w & f & w & w \\
  f & f & w & w & w \\
\end{tabular}
\end{center}
Für alle Kominationen aus A und B stimmt die Implikation und ist so mit tautologisch gültig.

\subsection{H1-2 Aussagenlogik im Alltag}
\subsubsection{Verneinung}

\begin{itemize}
  \item Es gibt mindestens einen Informatiker, der macht nicht gerne Mathematik.
  \item Alle Informatiker können programmieren oder lesen.
  \item Es regnet und die Straße ist nicht nass.
\end{itemize}

\subsubsection{}
Aus den Bedingung ergibt sich folgende Aussage für den potentiellen Mörder:
\((\neg V \land S) \lor (\neg N \land F)\)

\begin{center}
\begin{tabular}{|c || c c c c | c |}
  Erbe & V & S & N & F & \( (\neg V \land S) \lor (\neg N \land F) \) \\
  \hline
  Anton & w & w & f & f & f\\
  Birgit & w & f & f & w & w\\
  Carsten & f & f & w & w & f \\
\end{tabular}
\end{center}

Somit kommt nur Birgit als Mörderin in Frage.

\end{document}
