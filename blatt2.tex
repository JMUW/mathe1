\documentclass[a4paper,abstract,german]{scrreprt}

%
% Packages
%
\usepackage[german]{babel}
\usepackage[utf8]{inputenc}
\usepackage[T1]{fontenc}
\usepackage{lmodern}
\usepackage{amsmath}
\usepackage{bm}
\usepackage{amsmath}
\usepackage{amsfonts} 
\usepackage{amssymb}
\usepackage{amsthm}
\usepackage{graphicx}
\usepackage{enumitem}
\usepackage{tikz}
\usepackage[left=2.5cm,right=2.5cm,top=3cm,bottom=2cm,includeheadfoot]{geometry}

\setlength{\parindent}{0pt} 



%
% Eigene Macros
%
\newcommand{\N}{\mathbb{N}} 
\newcommand{\Z}{\mathbb{Z}} 
\newcommand{\Q}{\mathbb{Q}} 
\newcommand{\R}{\mathbb{R}} 
\newcommand{\C}{\mathbb{C}} 

%
% Beginn des eigentlichen Dokuments
%
\begin{document}


\noindent
{\begin{flushright}
%
% Hier die Namen einfüllen
%	
Name 1\\
Name 2\\
Name 3
\end{flushright}}
\begin{center}
	{\textbf  {Mathematik für Informatiker 1}} \\
	{\textbf {Abgabe 2. Übungsblatt}}
\end{center}\vspace{0.3cm}

%
% Aufzählung der Aufgaben
%
\begin{enumerate}
	
	
	\item[\textbf {H2.1}]
	\begin{enumerate}
	\item[\textbf {i)}]
    	De Morgansche Regeln in der Wahrheitstafel: \\
    	%
    	% Hier ist Platz für die Lösung einer Aufgabe
    	%
    	\begin{center}
        \begin{tabular}{||c c | c c | c c ||}
         A & B & $\neg (A \land B)$ & $\neg A \lor \neg B$ & $\neg (A \lor B)$ & $\neg A \land \neg B $ \\
         \hline
         w & w & f & f & w & w\\
         w & f & w & w & w & w\\
         f & w & w & w & w & w\\
         f & f & w & w & f & f\\
        \end{tabular}
        \end{center}
    \item[\textbf {ii)}]
        de Morgansche Regeln Beweis mengentheoretisch:

	\end{enumerate}
	\item[\textbf {H2.2}]
	Umschreiben der Aussagen unter der ausschließlichen Verwendung von $A, B, \uparrow$. Beweis, dass Aussage korrekt.
	\begin{enumerate}
	\item
    	$ A \land B$:\\
    	$(A \uparrow B) \uparrow(A \uparrow B)$
    \item
    	$ A \implies B$:\\
    \item
    	$ A \lor B$:\\
    \item
    	true(A):\\
    \item
    	false(A):\\

	\end{enumerate}
	
		\item[\textbf {H2.3}]
	Beweistechniken
	\begin{enumerate}
	\item
    	Seien $M,N, O$ Mengen mit $M \subset N$ und $N \subset O$, sowie $m \in M$.\\
    	
    	So gilt auch $m \in O$.
    \item
    	Beweise mit Kontraposition: $\forall n \in \mathbb{N}: n^2$ ungerade $\implies n$ ungerade  
    \item
        Beweise mit Widerspruchsbeweis: $\nexists z \in \mathbb{Z}: 0*z=1$\\
    	

	\end{enumerate}

\end{enumerate}

\end{document}
