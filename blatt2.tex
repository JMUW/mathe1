\documentclass[a4paper,abstract,german]{scrreprt}

%
% Packages
%
\usepackage[german]{babel}
\usepackage[utf8]{inputenc}
\usepackage[T1]{fontenc}
\usepackage{lmodern}
\usepackage{amsmath}
\usepackage{bm}
\usepackage{amsmath}
\usepackage{amsfonts} 
\usepackage{amssymb}
\usepackage{amsthm}
\usepackage{graphicx}
\usepackage{enumitem}
\usepackage{tikz}
\usepackage[left=2.5cm,right=2.5cm,top=3cm,bottom=2cm,includeheadfoot]{geometry}

\setlength{\parindent}{0pt} 



%
% Eigene Macros
%
\newcommand{\N}{\mathbb{N}} 
\newcommand{\Z}{\mathbb{Z}} 
\newcommand{\Q}{\mathbb{Q}} 
\newcommand{\R}{\mathbb{R}} 
\newcommand{\C}{\mathbb{C}} 

%
% Beginn des eigentlichen Dokuments
%
\begin{document}


\noindent
{\begin{flushright}
%
% Hier die Namen einfüllen
%	
\end{flushright}}
\begin{center}
	{\textbf  {Mathematik für Informatiker 1}} \\
	{\textbf {Abgabe 2. Übungsblatt}}
\end{center}\vspace{0.3cm}

%
% Aufzählung der Aufgaben
%
\begin{enumerate}
	
	
	\item[\textbf {H2.1}]
	\begin{enumerate}
	\item[\textbf {i)}]
    	De Morgansche Regeln in der Wahrheitstafel: \\
    	%
    	% Hier ist Platz für die Lösung einer Aufgabe
    	%
    	\begin{center}
        \begin{tabular}{||c c | c c | c c ||}
         A & B & $\overline{A \land B}$ & $\overline{A} \lor \overline{B}$ & $\overline{A \lor B}$ & $\overline{A} \land \overline{B} $ \\
         \hline
         w & w & f & f & w & w\\
         w & f & w & w & w & w\\
         f & w & w & w & w & w\\
         f & f & w & w & f & f\\
        \end{tabular}
        \end{center}
    \item[\textbf {ii)}]
    \textbf{z.Z. $\overline{M} \cap \overline{N} = \overline{M \cup N}$}
    
        Seien $x \in X$, $M \subset X$ und $N\subset X$.
        
        Dann sind die Mengen $\overline{M} \cap \overline{N}$ und $\overline{M \cup N}$ definiert durch:
        
        $\overline{M} \cap \overline{N} := \{ x |x \notin M \land x \notin N \}$
        
        $\overline{M \cup N} := \{x|\overline{x \in M \lor x \in N} \}$  \\
        
        Wir formen um gemäß de Morganschen Regel $\overline{A\lor B} \iff \overline{A} \land \overline{B}  $:
        
       $\overline{M} \cap \overline{N} := \{ x |x \notin M \land x \notin N \}$ \\ 
        
         Somit ist gezeigt, dass $\overline{M} \cap \overline{N} = \overline{M \cup N}$: auch mengentheoretisch äquivalent.
         
         \bigskip
         \textbf{z.Z.
         $\overline{M \cap N} = \overline{M} \cup \overline{N}$:}
         
         Es gelte allgemein $x\in X$, $M \subset X$ und $N \subset X$.
         
         $\overline{M \cap N} := \{x| \overline{x \in M \land x \in N} \}$ nach de Morgansche Regel aus i) ist das gleichbedeutend mit $\{x| x \notin M \lor x \notin N \}$. 
         
        $\overline{M} \cup \overline{N} := \{x|x \notin M \lor x \notin N\}$ und ist somit identisch zu der umgeformten Form von $\overline{M \cap N}$ und  $\overline{M \cap N} = \overline{M} \cup \overline{N}$ somit äquivalent.

	\end{enumerate}
	\item[\textbf {H2.2}]
	Umschreiben der Aussagen unter der ausschließlichen Verwendung von $A, B, \uparrow$. Beweis, dass Aussage korrekt.
	\begin{enumerate}
	\item $ A \land B \iff (A \uparrow B) \uparrow(A \uparrow B)$
	
	\begin{center}
        \begin{tabular}{||c c | c | c | c ||}
         A & B & $ A \land B$ & $A \uparrow B$ & $(A \uparrow B) \uparrow(A \uparrow B)$  \\
         \hline
         w & w & w & f & w \\
         w & f & f & w & f \\
         f & w & f & w & f \\
         f & f & f & w & f \\
        \end{tabular}
        \end{center}
    	
    	
    \item
    	$ A \rightarrow B \iff A \uparrow \overline{B} \iff A \uparrow (B \uparrow B) $ \\
    		\begin{center}
        \begin{tabular}{||c c | c | c ||}
         A & B & $ A \rightarrow B$ & $A \uparrow (B \uparrow B)$  \\
         \hline
         w & w & w & w \\
         w & f & f & f \\
         f & w & w & w \\
         f & f & w & w  \\
        \end{tabular}
        \end{center}
        
    \item
    	$ A \lor B \iff \overline{A} \uparrow \overline{B}$
    	$\iff  (A \uparrow A) \uparrow (B \uparrow B)$
    	\begin{center}
    	\begin{tabular}{||c c | c | c ||}
         A & B & $ A \lor B$ & $(A \uparrow A) \uparrow (B \uparrow B)$ \\
         \hline
         w & w & w & w\\
         w & f & w & w\\
         f & w & w & w\\
         f & f & f & f\\
        \end{tabular}
        \end{center}
        
    \item
    	true(A):\\
    	$\iff A \lor \overline{A} \iff  \overline{A} \uparrow \overline{\overline{A}} \iff  \overline{A} \uparrow A \iff  (A \uparrow A) \uparrow A$
    	\begin{center}
    	\begin{tabular}{||c | c | c | c  ||}
         A & $A \uparrow A$  & $ (A \uparrow A) \uparrow A$ & true(A) \\
         \hline
         w & f & w & w\\
         f & w & w & w\\
        \end{tabular}
        \end{center}
        
    \item
    	false(A):\\
    	Negierung von true(A), bzw.
    	$A \land \overline{A} \iff (A \uparrow (A \uparrow A)) \uparrow(A \uparrow (A \uparrow A))$
    	
    	\begin{center}
    	\begin{tabular}{||c | c | c | c  ||}
         A & $ (A \uparrow A) \uparrow A$ & $[(A \uparrow A) \uparrow A)] \uparrow [(A \uparrow A) \uparrow A)]$ & false(A) \\
         \hline
         w & w & f & f\\
         f & w & f & f\\
        \end{tabular}
        \end{center}

	\end{enumerate}
	
		\item[\textbf {H2.3}]
	Beweistechniken
	\begin{enumerate}
	\item
    	Seien $M,N, O$ Mengen mit $M \subset N$ und $N \subset O$, sowie $m \in M$.\\
    	
    	M ist eine Teilmenge von N und daher nach Definition der Teilmenge $(x \in M \implies x \in N)$ $m \in N$, da $m \in M$.
    	
    	Da N wiederum selbst eine Teilmenge von O ist, so gilt auch hier nach Def. $(x \in N \implies x \in O)$ durch $m \in N$ auch $m \in O$.
    
    \item
    	Z.z das gilt $n \in \mathbb{N}: n^2$ ungerade $\implies n$ ungerade\\  
    	Nach Kontraposition gelte somit $ n \in \mathbb{N}: n$ gerade $\implies n^2$ gerade.
    	
    	Da n gerade, also $n \in 2\Z$ ist, so gilt $n = 2*k$, wobei $k \in \N $:
    	
    	Das Quadrat dieser gerade Zahl n beträgt $ n^2 = (2k)^2 = 4*k^2 $
    	
    	Durch den Teiler $4$ wird das Produkt immer gerade sein. Somit wäre die Kontraposition bewiesen, dass das Quadrat einer geraden Zahl immer gerade ist.
    	
    	Somit gelte auch die ursprüngliche Aussage, dass das ungerade Quadrat durch eine ungerade Basis entstand.
    	
    \item
        Beweise mit Widerspruchsbeweis: $\nexists z \in \Z: 0*z=1$\\

    	Annahme: Es gibt eine ganze Zahl $z \in \Z$, für die $0*z=1$ gilt:
    	
    	$\exists z \in \Z: 0*z=1$
    	
    	$0*z=1 \iff z = \frac{1}{0} \implies $ n.def. Widerspruch!
    	
    	
    	Da $\frac{1}{0}$ nicht definiert ist existiert keine Lösung für $z \in Z$, also $\nexists z \in \Z: 0*z=1$.

	\end{enumerate}

\end{enumerate}
